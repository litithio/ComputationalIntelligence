\section{Gen-String \& Encoding}

\paragraph{Gen-String \& Encoding}
Wir optimieren die drei Mahlzeitenfaktoren (\IKRtxtpl):
\[
x \;=\; (\ICRf,\ \ICRm,\ \ICRa)^\top \in \mathbb{R}^3,
\]
wobei \(\mathrm{ICR}_j\) in \([\text{g KH} / \text{U Insulin}]\) gemessen wird und die Box-Grenzen
\[
5 \;\le\; \mathrm{ICR}_j \;\le\; 30 \qquad \text{für } j \in \{f,m,a\}
\]

gelten.

\paragraph{Begründung der Box-Grenzen (5--30 g/U)}
Die Schranken wirken \emph{auf den Faktor} (Verhältnis in g/U), nicht auf die Bolusdosis. Sie dienen drei Zielen:
\begin{itemize}
  \item \textbf{Klinische Plausibilität.} Bei Erwachsenen liegen typische ICRs häufig im Bereich 10--20 g/U; morgens sind stärkere Faktoren (5--12 g/U) üblich, abends eher 12--20 g/U. Werte weit außerhalb von 5--30 g/U sind selten und deuten meist auf Sondersituationen (Infekt, Steroide, ausgeprägte Insulinresistenz) hin.
  \item \textbf{Sicherheit und numerische Stabilität.} Sehr kleine ICRs (\(<5\) g/U) führen bei üblichen Mahlzeiten zu sehr großen Bolusdosen (Hypo-Risiko), sehr große ICRs (\(>30\) g/U) zu praktisch keiner Abdeckung (Hyper-Risiko). Die Box verhindert degenerierte Lösungen des Optimierers.
  \item \textbf{Interpretierbarkeit und Suchraumbegrenzung.} Der Suchraum bleibt kompakt und die Gewichte der Zielfunktion stabil kalibrierbar. Die Grenzen sind \emph{patientenspezifisch anpassbar} (z.\,B. 6--24 g/U), falls klinische Daten abweichende Bereiche nahelegen.
\end{itemize}
Die Bolusdosis zur Mahlzeit ergibt sich weiterhin aus \(B_{\mathrm{meal}}(t)=C(t)/\ICR(t)\) und wird \emph{nicht} direkt beschränkt; die Box-Grenzen betreffen ausschließlich die Entscheidungsvariablen \(x=(\ICRf,\ICRm,\ICRa)\).


\paragraph{Semantik des Gen-Strings}
Jede Komponente von \(x=(\ICRf,\ICRm,\ICRa)\) steht für das \IKRtxt{} der jeweiligen Mahlzeitenperiode (Früh, Mittag, Abend). Der Gen\,\(\to\)\,Phänotyp-\emph{Mapping} erfolgt \emph{zeitbasiert}: Die in Abschnitt \emph{Simulationsregeln des digitalen Zwillings} definierten Fenster \(\mathcal W_{\mathrm{Morgen}},\mathcal W_{\mathrm{Mittag}},\mathcal W_{\mathrm{Abend}}\) bestimmen, welcher ICR-Wert zu einem Zeitpunkt \(t\) gilt; formal \(\ICR(t)\in\{\ICRf,\ICRm,\ICRa\}\) mit der dort angegebenen Stückweise-Definition. Damit ist klar nachvollziehbar, welcher Genwert welches Tageszeitfenster steuert.

\paragraph{Warum diese Kodierung?}
Die dreidimensionale Kodierung ist (i) \emph{klinisch interpretierbar} und entspricht gelebter Praxis, (ii) \emph{identifizierbar} und stabil gegen\"uber Rauschen (kleiner Suchraum, weniger Overfitting), (iii) \emph{kompatibel} mit der Quartalslogik: \(x\) bleibt im Quartal konstant und wird nur an Quartalsgrenzen angepasst (vgl. Schrittbegrenzung). Eine explizite Optimierung von Boluszeitpunkten oder Basalprofilen wird bewusst vermieden, um die Komplexit\"at niedrig zu halten.

\paragraph{Bolusberechnung im Simulator (zur Einordnung).}
Die Mahlzeitenbolusgabe richtet sich nach den angek\"undigten Kohlenhydraten \(C(t)\) [g] zum Zeitpunkt \(t\) und dem aktiven Verh\"altnis: \[B_{\mathrm{meal}}(t)\;=\;\frac{C(t)}{\ICR(t)}.\] Korrekturboli (falls im Zwilling aktiv) sind \emph{nicht} Teil der Entscheidungsvariable und folgen einer fixen Regel; sie beeinflussen \(x\) nicht und werden in den Sicherheitskriterien (TBR/TAR) indirekt bewertet.


\paragraph{Mutationsoperatoren}
Wir verwenden zwei Mutationsoperatoren für die ICR-Gene. Beide respektieren die Box-Grenzen \([5,30]\) und die Schrittbegrenzung pro Runde (vgl. Abschnitt \emph{Nebenbedingungen auf der Entscheidungsvariable}).
\begin{itemize}
  \item[\textbf{G1}] \textbf{Einzel-Gen (Gauß):} Wähle \(j\in\{f,m,a\}\) gleichverteilt. Ziehe \(\delta \sim \mathcal N(0,\sigmaloc^{2})\), trunkiert auf \([-\DICR,\DICR]\). Setze \(\ICR'_j=\ICR_j+\delta\) und lasse die anderen Gene unverändert; projiziere dann \(x'\) komponentenweise auf \([5,30]\).
  \item[\textbf{G2}] \textbf{Gemeinsamer Shift (Gauß):} Ziehe \(\delta \sim \mathcal N(0,\sigmaall^{2})\), trunkiert auf \([-\DICR,\DICR]\). Setze \(x' = x + \delta\,\mathbf{1}\) und projiziere auf \([5,30]^3\).
\end{itemize}
Startwerte: \(\sigmaloc=0{,}3\,\mathrm{g/U}\), \(\sigmaall=0{,}2\,\mathrm{g/U}\).
\(\rho_{\mathrm{mut}}=0.3\); die Auswahlwahrscheinlichkeiten sind \(p(\text{G1})=0.8\), \(p(\text{G2})=0.2\).\\
\emph{Notation:} \(\Pi_{[a,b]}\) bezeichnet die Projektion auf das Intervall \([a,b]\); \(\Pi_{[5,30]^3}\) ist die komponentenweise Projektion auf das kartesische Produkt.

\paragraph{Warum Gauß-Mutationen?}
Wir verwenden Gauß‑Mutationen, d.\,h.\ Änderungen sind zentriert um \(0\) und kleine Schritte treten viel häufiger auf als große. Das spiegelt die klinische Realität vorsichtiger, iterativer Feinanpassungen wider und reduziert unplausible Sprünge. Durch Trunkierung auf \([-\DICR,\DICR]\) und anschließendes Clipping auf \([5,30]\) bleiben Sicherheitsgrenzen gewahrt.

\paragraph{Klinische Begründung für den gemeinsamen Shift}
Der gemeinsame Shift (alle drei ICRs um denselben kleinen Betrag) bildet reale, systemweite Veränderungen des Insulinbedarfs ab. Beispiele: akute Infekte, Fieber, Stress oder die Gabe von Kortikosteroiden erhöhen den Bedarf (alle ICRs verschieben sich nach unten = mehr Insulin pro g KH); mehr Bewegung, Gewichtsverlust oder eine Zunahme der Insulinsensitivität senken ihn (alle ICRs nach oben). Auch Daumenregeln wie die 500/450‑Regel koppeln ICR grob an die Gesamttagesdosis; ändert sich diese global, verschieben sich die ICRs konsistent. Der Shift verändert somit das \emph{Niveau}, während der Einzel‑Gen‑Jitter die \emph{Form} (Früh vs.\ Mittag vs.\ Abend) justiert. Trunkierung/Clipping stellt sicher, dass klinische Grenzen (\([5,30]\) g/U, \(\pm\DICR\)) eingehalten werden.

\paragraph{Hinweis zur Homogenitätsstrafe}
Der gemeinsame Kleinst-Shift (G2) ist \emph{maßstabsnah} und ändert das Verhältnis \(r(x)=\tfrac{\max\{\ICRf,\ICRm,\ICRa\}}{\min\{\ICRf,\ICRm,\ICRa\}}\) kaum; damit bleibt der optionale Verhältnis-Guardrail \(P_{\mathrm{disp}}(x)\) in der Regel unverändert. Der Einzel-Gen-Schritt (G1) kann \(r(x)\) vergrößern; überschreitet er die Toleranz \(\rmax\), greift die weiche Strafe (vgl. Abschnitt \emph{Nebenbedingungen auf der Entscheidungsvariable}).

\paragraph{Gaussian\-Batch\-Proposals}
Pro Iteration erzeugen wir einen stochastischen \emph{Batch} von $B$ Kandidaten um den aktuellen Elternvektor $x$, basierend auf den Gau{\ss}\-Mutationen aus Abschnitt \emph{Mutationsoperatoren}:
\begin{itemize}
  \item Ziehe f\"ur jedes der $B$ Elemente unabh\"angig einen Operator: G1 (Einzel\-Gen) mit Wahrscheinlichkeit $p(\mathrm{G1})$ oder G2 (gemeinsamer Shift) mit $p(\mathrm{G2})$ (Standard: $0{,}8/0{,}2$).
  \item G1: W\"ahle $j\in\{f,m,a\}$ gleichverteilt, ziehe $\delta\sim\mathcal N(0,\sigma_{\text{loc}}^2)$, trunkiert auf $[-\Delta_{\mathrm{ICR}},\Delta_{\mathrm{ICR}}]$, setze $\ICR'_j=\ICR_j+\delta$.
  \item G2: Ziehe $\delta\sim\mathcal N(0,\sigma_{\text{all}}^2)$, trunkiert auf $[-\Delta_{\mathrm{ICR}},\Delta_{\mathrm{ICR}}]$, setze $x'=x+\delta\,\mathbf 1$.
  \item Projiziere stets komponentenweise auf die Box $[5,30]^3$ (Clippen), so dass alle Vorschl\"age zul\"assig bleiben und die Schrittbegrenzung je Runde eingehalten ist.
\end{itemize}
Damit entf\"allt die fr\"uhere deterministische Nachbarschaft $\mathcal N(x)$; kleine, klinisch plausible Schritte entstehen nat\"urlich durch die Gau{\ss}\-Kerne.

\paragraph{Schnell\-Screening vor der Simulation.}
Vorschl\"age mit zu starker Spreizung der ICRs werden bereits \emph{vor} der teuren Simulation verworfen: Falls der harte Verh\"altnis\-Screen $r(x')>r_{\text{screen}}$ (z.\,B. $r_{\text{screen}}=2.0$), wird der Kandidat gedroppt. Der optionale weiche Guardrail $P_{\mathrm{disp}}$ wirkt weiterhin im zul\"assigen Bereich $r(x')\le r_{\text{screen}}$. Wir simulieren anschlie\ss end h\"ochstens die verbleibenden $B$ Vorschl\"age dieser Iteration.

\paragraph{Auswahlregel pro Iteration.}
(1) Gau{\ss}\-Batch $\mathcal B(x)$ mit $B$ Vorschl\"agen erzeugen; (2) nach Schnell\-Screening alle zul\"assigen Kandidaten simulieren und bewerten; (3) \emph{best\-improving} \"ubernimmt den Elternplatz. F\"allt keine Verbesserung an, erlauben wir mit kleiner Wahrscheinlichkeit (z.\,B. 0.2) einen zus\"atzlichen zuf\"alligen G1\-Kick mit leicht erh\"ohter Varianz, um lokale Plateaus zu verlassen. Dieser Mix aus stochastisch kleinen Schritten und sparsamer Exploration h\"alt die Suche effizient und klinisch plausibel.